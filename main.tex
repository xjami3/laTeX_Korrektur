\documentclass[12pt,a4paper,oneside]{article} % oneside für einseitigen Druck wie im Skript

% GRUNDEINSTELLUNGEN
\usepackage[utf8]{inputenc}
\usepackage[T1]{fontenc}
\usepackage[ngerman]{babel}
\usepackage{lmodern} % Bessere Schriftqualität als Standard CM
\usepackage{textgreek} % Für griechische Buchstaben wie β

% SEITENGEOMETRIE (gemäß Vorgabe: oben 4, unten 2, links 4, rechts 2)
\usepackage[left=4cm,right=2cm,top=4cm,bottom=2cm,headheight=14.5pt]{geometry}

% KOPF- UND FUSSZEILE (Seitenzahl oben rechts)
\usepackage{fancyhdr}
\pagestyle{fancy}
\fancyhf{} % Alle Felder leeren
\rhead{\thepage} % Seitenzahl oben rechts
\renewcommand{\headrulewidth}{0pt} % Keine Linie unter der Kopfzeile

% ZEILENABSTAND (Text 1.5, Fußnoten etc. 1.0)
\usepackage{setspace}
\onehalfspacing % Für den Haupttext

% FUSSNOTEN
\usepackage{footmisc} % Für bessere Kontrolle über Fußnoten
\renewcommand{\footnotelayout}{\setstretch{1.0}\footnotesize} % Einfacher Zeilenabstand, 10pt

% ABSATZEINZUG UND -ABSTAND
\setlength{\parindent}{0pt} % Kein Einzug für den ersten Absatz
\setlength{\parskip}{6pt plus 2pt minus 2pt} % Abstand zwischen Absätzen

% TYPOGRAPHIE VERBESSERUNGEN
\usepackage{microtype} % Bessere Randausgleich etc.
\usepackage[autostyle, german=quotes]{csquotes} % Korrekte deutsche Anführungszeichen
\MakeOuterQuote{"}

% GRAFIKEN
\usepackage{graphicx}
\graphicspath{ {./bilder/} } % Optional: Unterordner für Bilder

% TABELLEN
\usepackage{booktabs} % Schönere Tabellenlinien
\usepackage{caption}
\captionsetup[figure]{name=Abb.,labelsep=colon,skip=10pt}
\captionsetup[table]{name=Tab.,labelsep=colon,skip=10pt,position=top} % Tabellenbeschriftung oben

% LITERATURVERZEICHNIS (BibLaTeX)
\usepackage[
    backend=biber,
    style=authortitle-ibid, % Zitiertitel und ibid. für Folgezitate
    sorting=nyt, % Sortierung nach Name, Jahr, Titel
    giveninits=true, % Vornamen als Initialen
    uniquename=init, 
    maxbibnames=3, % Ab 3 Autoren "et al." im Verzeichnis
    maxcitenames=2, % Ab 2 Autoren "et al." im Zitat
    doi=false, isbn=false, url=true, eprint=false % url=true ist wichtig für die Anzeige von URLs
]{biblatex}
\addbibresource{literatur_template.bib} % Name deiner .bib Datei
\DeclareFieldFormat{bibentrysetcount}{\mkbibparens{\printtext[bibentrysetcount]{#1}}} % Für Zählungen
\setlength\bibitemsep{0.5\baselineskip} % Abstand zwischen Bib-Einträgen (einfach)

% HYPERLINKS (für das Inhaltsverzeichnis)
\usepackage[bookmarks=true, hidelinks]{hyperref}

% HILFSPAKET FÜR LOREM IPSUM
\usepackage{lipsum}

% Eigene Definition für Quellenangabe unter Abbildungen/Tabellen
\newcommand{\source}[1]{\par\vspace{2pt}\noindent\footnotesize Quelle: #1}
\newcommand{\ownsource}{\source{Eigene Darstellung}}

%----------------------------------------------------------------------------------------
%   DOKUMENTENBEGINN
%----------------------------------------------------------------------------------------
\begin{document}

%----------------------------------------------------------------------------------------
%   TITELSEITE
%----------------------------------------------------------------------------------------
\begin{titlepage}
    \centering
    \vspace*{2cm} % Abstand von oben

    {\huge\bfseries ?(Kritische) Analyse der Einführung von Microsoft CoPilot (ableitung für erfolgsfaktoren von KI assistenten)?}

    \vspace{2.5cm}

    {\Large Projektarbeit 2} % Oder Bachelorarbeit etc.

    \vspace{0.5cm}

    {\normalsize Studienjahrgang 2023/2024\\ % Beispiel
    Kurs WI23B\\ % Beispiel
    Fakultät Wirtschaft\\
    Studiengang Wirtschaftsinformatik}

    \vspace{1cm}

    {\Large Duale Hochschule Baden-Württemberg Villingen-Schwenningen}

    \vspace{2.5cm}

    \begin{tabular}{@{}ll}
        Bearbeiter: & Jamie Dave Adler \\
        Ausbildungsbetrieb: & Karl Storz SE \& Co. KG \\
        Betreuende Dozentin: & Prof. Dr. Andreas Bildstein \\ % Beispiel aus Vorlage
    \end{tabular}

    \vfill % Füllt den Rest der Seite nach unten
\end{titlepage}
\pagenumbering{Roman} % Römische Seitenzahlen für Verzeichnisse etc.
\setcounter{page}{1} % Deckblatt ist Seite I (wird nicht gedruckt)
\thispagestyle{empty} % Keine Seitenzahl auf Deckblatt

%----------------------------------------------------------------------------------------
%   SPERRVERMERK (falls benötigt) - Eigene Seite ohne Paginierung
%----------------------------------------------------------------------------------------
\newpage
\thispagestyle{empty} % Keine Seitenzahl auf Deckblatt
\vspace*{4cm} % Etwas Abstand von oben
\noindent
Der Inhalt dieser Arbeit darf weder als Ganzes noch in Auszügen Personen außerhalb des Prüfungs- und Evaluationsverfahrens zugänglich gemacht werden, sofern keine anders lautende Genehmigung des Dualen Partners vorliegt.
\clearpage

%----------------------------------------------------------------------------------------
%   INHALTSVERZEICHNIS
%----------------------------------------------------------------------------------------
\newpage
\setcounter{page}{2} % Inhaltsverzeichnis ist Seite II (wird nicht gedruckt)
\thispagestyle{empty} % Keine Seitenzahl auf Inhaltsverzeichnis
\tableofcontents
\clearpage

%----------------------------------------------------------------------------------------
%   ABKÜRZUNGSVERZEICHNIS (optional)
%----------------------------------------------------------------------------------------
% \newpage % Jedes Verzeichnis auf neuer Seite
% \section*{Abkürzungsverzeichnis}
% \addcontentsline{toc}{section}{Abkürzungsverzeichnis} % Ins Inhaltsverzeichnis
% \begin{tabular}{@{}ll}
%     ABC & Abkürzung Bedeutung \\
%     XYZ & Weitere Abkürzung \\
% \end{tabular}
% \clearpage

%----------------------------------------------------------------------------------------
%   ABBILDUNGSVERZEICHNIS
%----------------------------------------------------------------------------------------
\newpage
\listoffigures
\addcontentsline{toc}{section}{Abbildungsverzeichnis}
\clearpage

%----------------------------------------------------------------------------------------
%   TABELLENVERZEICHNIS
%----------------------------------------------------------------------------------------
\newpage
\listoftables
\addcontentsline{toc}{section}{Tabellenverzeichnis}
\clearpage

%----------------------------------------------------------------------------------------
%   HAUPTTEIL
%----------------------------------------------------------------------------------------
\newpage
\pagenumbering{arabic} % Arabische Seitenzahlen für den Hauptteil
\setcounter{page}{1} % Beginnt mit Seite 1

\section{Einleitung}

\subsection{Relevanz und Forschungslücke}
Die Einführung generativer KI in die Wissensarbeit gilt als tiefgreifende sozio-technische Transformation. Werkzeuge wie Microsoft Copilot versprechen eine signifikante Steigerung der Produktivität und Arbeitsqualität. Frühe empirische Studien untermauern diese hohen Erwartungen mit quantitativen Belegen: Nutzer berichten von einer um 29\,\% höheren Geschwindigkeit bei Standardaufgaben, einer täglichen Zeitersparnis von durchschnittlich 14 Minuten und einer Verbesserung der Arbeitsqualität\footcite{Microsoft2025Worklab}.

Trotz dieses evidenten Potenzials ist die erfolgreiche Implementierung kein Selbstläufer. Die bloße Bereitstellung der Technologie führt nicht automatisch zu deren Nutzung oder zur Realisierung der versprochenen Vorteile. In der Praxis zeigt sich eine erhebliche Lücke zwischen dem technologischen Potenzial und der tatsächlichen Adoption durch die Mitarbeiter. Diese Lücke wird durch menschliche Faktoren wie mangelndes Vertrauen, hohe Aufwandserwartungen und die Komplexität der neuen Interaktionsform des "Prompting"\footcite{Arxiv2025BeyondTraining} sowie durch organisationale Versäumnisse bei der Steuerung des Wandels und der Schaffung technischer Grundvoraussetzungen\footcite{MicrosoftLearn2025CopilotSecurity} verursacht.

Hieraus ergibt sich eine Forschungslücke: Während die technologischen Möglichkeiten und die hohen Produktivitätserwartungen intensiv diskutiert werden, sind empirisch fundierte und modellgebundene Handlungsempfehlungen für den \textit{Enterprise}-Kontext, die diese sozio-technischen Aspekte integriert betrachten, noch rar. Der Beitrag dieser Arbeit liegt daher in einer theoriegeleiteten, kritischen Analyse der Implementierung, die über eine chronologische Projektbeschreibung hinausgeht und stattdessen die komplexen Wechselwirkungen zwischen Technologie, Organisation und Mensch untersucht.

\subsection{Problemstellung}
Die Kernherausforderung bei der Einführung generativer KI-Assistenten liegt in der Orchestrierung der verschiedenen Dimensionen des Wandels. Wie lassen sich Einführungen generativer KI so gestalten, dass \textbf{technologische}, \textbf{organisationale} und \textbf{menschliche} Faktoren (T–O–M) aufeinander abgestimmt wirken und die beabsichtigten Nutzeneffekte tatsächlich materialisieren?

Diese übergreifende Problemstellung manifestiert sich in spezifischen Teilproblemen:
\begin{itemize}
    \item \textbf{Technologisch (T):} Die Technologie wird oft in eine bestehende IT-Landschaft implementiert, deren "Tech Hygiene" (z.\,B. Datenklassifizierung, Zugriffsrechte) unzureichend ist. Dies führt nicht nur zu Sicherheitsrisiken wie Datenlecks\footcite{MicrosoftLearn2025CopilotSecurity}, sondern auch zu einer schlechten Nutzererfahrung durch ungenaue oder irrelevante Ergebnisse ("Garbage In, Garbage Out").
    \item \textbf{Menschlich (M):} Die Nutzer werden mit einer hohen \textit{Aufwandserwartung} (Effort Expectancy) konfrontiert. Sie müssen eine neue Kompetenz, die "Prompt Literacy", entwickeln\footcite{Monash2025PromptLiteracy}, scheitern aber oft anfangs an der Komplexität, was zu Frustration und schneller Abkehr vom Werkzeug führt\footcite{Arxiv2025BeyondTraining}. Gleichzeitig bestehen signifikante psychologische Barrieren wie die Angst vor Arbeitsplatzverlust oder Überwachung\footcite{CyberSecurityIntel2025EmployeeResistance}.
    \item \textbf{Organisational (O):} Den Organisationen fehlt es häufig an einer klaren strategischen Vision\footcite{Prosci2025KotterTheory}, einer adäquaten Governance-Struktur (z.\,B. einem AI Center of Excellence)\footcite{Dataiku2025CoEBestPractices} und effektiven Change-Mechanismen, um die technologischen Voraussetzungen zu schaffen und die Mitarbeiter im Lernprozess zu begleiten.
\end{itemize}
Ein Scheitern der Adoption ist oft das Resultat einer Fehlausrichtung dieser drei Dimensionen.

\subsection{Zielsetzung und Beitrag der Arbeit}
Das primäre Ziel dieser Arbeit ist die kritische Analyse der Erfolgsfaktoren und Barrieren bei der Implementierung von Microsoft Copilot. Die Arbeit prüft und konkretisiert etablierte Modelle der Akzeptanzforschung (TAM/UTAUT)\footcite{UTAUT2_Students_Cedtech2025} und des Change-Managements (Kotter)\footcite{Prosci2025KotterTheory} im spezifischen Kontext generativer KI.

Als \textit{Fallbeleg} dient die Einführung von Microsoft Copilot bei der \emph{Karl Storz SE \& Co. KG (KS)}. Diese Fallstudie wird genutzt, um die theoretischen Modelle empirisch zu fundieren und ihre Wechselwirkungen zu illustrieren. Die Arbeit liefert explizit keine chronologische Nacherzählung des Rollout-Projekts.

Der wissenschaftliche Beitrag der Arbeit besteht darin, die kritischen \textit{Interdependenzen} zwischen der technologischen (T), organisationalen (O) und menschlichen (M) Dimension aufzuzeigen. Es wird analysiert, wie Defizite in einer Dimension (z.\,B. mangelnde "Tech Hygiene" (T)) direkt auf eine andere Dimension (z.\,B. reduzierte "Leistungserwartung" (M)) durchschlagen. Basierend auf dieser kritischen Analyse werden verallgemeinerbare Designprinzipien und Handlungsempfehlungen für die Praxis abgeleitet.

\subsection{Forschungsfragen und Hypothesen}
Zur Erreichung der Zielsetzung werden die folgenden Forschungsfragen (RQ) im Rahmen des T–O–M-Modells untersucht:

\begin{itemize}
    \item \textbf{RQ1 (Technologie):} Welche technischen Voraussetzungen und welche Maßnahmen der "Tech Hygiene" (z.\,B. Datenklassifizierung, Zugriffsmanagement) sind notwendig, um die Adoption durch die Nutzer zu ermöglichen und Compliance-Risiken zu minimieren?
    \item \textbf{RQ2 (Organisation):} Welche Governance-Strukturen (z.\,B. CoE) und Change-Management-Mechanismen (z.\,B. Sponsoring, Champions-Programme) korrelieren mit einer höheren Nutzungsintensität und der Überwindung von Widerständen?
    \item \textbf{RQ3 (Mensch):} Wie beeinflussen die Faktoren des UTAUT-Modells – insbesondere Erwartungsmanagement (PE/EE), soziales Umfeld (SI) und unterstützende Maßnahmen (FC) wie "Prompt-Literacy"-Training – die Nutzungsabsicht (\textit{Behavioral Intention}) und die tatsächliche Nutzung?
\end{itemize}

Basierend auf der vorab durchgeführten Recherche (siehe Kap. 2) werden folgende Hypothesen zur Überprüfung im Rahmen der Fallstudie formuliert:

\begin{itemize}
    \item \textbf{H1 (M):} Eine hohe \textit{Aufwandserwartung} (EE), bedingt durch eine wahrgenommene Komplexität des Promptings, korreliert negativ mit der Nutzungsabsicht, selbst wenn die \textit{Leistungserwartung} (PE) hoch ist.
    \item \textbf{H2 (T/M):} Defizite in der "Tech Hygiene" (T) führen zu unzuverlässigen KI-Ergebnissen und korrelieren daher negativ mit der \textit{Leistungserwartung} (M) der Nutzer.
    \item \textbf{H3 (O/M):} "Sozialer Einfluss" (SI), manifestiert durch die sichtbare Nutzung durch Führungskräfte und "KI-Champions", sowie dedizierte "unterstützende Rahmenbedingungen" (FC) in Form von Trainings (Prompt Literacy) steigern die Nutzungsabsicht signifikant.
\end{itemize}

\subsection{Methodisches Vorgehen und Aufbau der Arbeit}
Um die Forschungsfragen zu beantworten, wird ein qualitativ dominiertes \textit{Embedded Single-Case-Study-Design} (Fallstudie) mit Mixed-Methods-Elementen angewandt. Der Fall "Karl Storz" dient der theoriegeleiteten Erklärung und analytischen Verallgemeinerung, nicht der statistischen Populationsschätzung. Die Datenerhebung stützt sich auf die Triangulation von (1) quantitativen Nutzungs-KPIs, (2) qualitativen, leitfadengestützten Interviews mit Stakeholdern (Projektleitung, IT, Endnutzer) und (3) der Analyse von Projektdokumenten (Governance-Artefakte, Trainingsunterlagen).

Die Arbeit ist wie folgt strukturiert: \textbf{Kapitel 2} legt den theoretischen Rahmen dar, indem es den Stand der Forschung zu KI-Assistenten, die Akzeptanzmodelle (UTAUT) und Change-Modelle (Kotter) sowie die technologischen Governance-Aspekte (Responsible AI) detailliert. \textbf{Kapitel 3} beschreibt das Forschungsdesign, die Datenerhebung und die Auswertungsmethodik. \textbf{Kapitel 4} präsentiert die Ergebnisse der Fallstudie, strukturiert entlang der drei Dimensionen des T–O–M-Frameworks. \textbf{Kapitel 5} diskutiert diese Ergebnisse, gleicht sie mit der Theorie ab und leitet die generalisierbaren Designprinzipien ab, bevor \textbf{Kapitel 6} die Arbeit mit einem Fazit und Handlungsempfehlungen abschließt.


% ----------------------------------------
\section{Theoretischer Rahmen und Stand der Forschung}

Dieses Kapitel legt das theoretische Fundament ("Soll-Zustand") für die nachfolgende kritische Analyse der Copilot-Einführung. Es definiert die zentralen Konzepte und Modelle, die als analytische "Messlatte" für die empirische Untersuchung in Kapitel 4 dienen. Zunächst wird der Status Quo von Enterprise-KI-Assistenten beleuchtet (Kap.~2.1). Darauf aufbauend werden die Theorien der Technologieakzeptanz (UTAUT) für die menschliche Dimension (Kap.~2.2) und die Modelle des Change Managements (Kotter) für die organisationale Dimension (Kap.~2.3) vorgestellt. Anschließend werden die technologischen Governance-Aspekte (Kap.~2.4) detailliert. Das Kapitel schließt mit der Synthese dieser Elemente zu einem integrierten T–O–M-Bezugsrahmenmodell (Kap.~2.5).

\subsection{State of the Art zu Enterprise-AI und generativen KI-Assistenten}
Generative KI-Assistenten, tief integriert in die täglichen Arbeitsabläufe ("Enterprise-AI"), versprechen eine signifikante Transformation der Wissensarbeit. Im Zentrum steht das Nutzenversprechen einer messbaren Produktivitäts- und Qualitätssteigerung.

\paragraph{Nutzenpotenziale}
Frühe empirische Studien, insbesondere von Microsoft zur Einführung von Copilot, quantifizieren diesen Nutzen. Nutzer berichten von einer um 29\,\% höheren Geschwindigkeit bei Standardaufgaben wie Suchen, Schreiben und Zusammenfassen. Die durchschnittliche tägliche Zeitersparnis wird auf 14 Minuten beziffert, wobei 68\,\% der Nutzer eine Verbesserung der Arbeitsqualität und 57\,\% eine Steigerung der eigenen Kreativität angaben\footcite{Microsoft2025Worklab}. Bei Entwicklern, die GitHub Copilot nutzen, wurde die Aufgabenerledigung sogar um über 55\,\% beschleunigt\footcite{Arxiv2023GitHubImpact}. Diese datengestützte "Leistungserwartung" ist der primäre Treiber für die Investitionsbereitschaft von Unternehmen.

\paragraph{Risiken und Herausforderungen}
Dem Nutzen stehen erhebliche Risiken gegenüber. Das prominenteste technologische Risiko sind "Halluzinationen" – die Generierung faktisch falscher, aber plausibel klingender Ausgaben. Dies führt zu einem fundamentalen Mangel an Vertrauen in "Blackbox"-Modelle, deren Entscheidungsprozesse für den Nutzer nicht nachvollziehbar sind\footcite{CyberSecurityIntel2025EmployeeResistance}.

Das größte \textit{Compliance}-Risiko liegt jedoch nicht in der KI selbst, sondern im Zustand der bestehenden Datenlandschaft. In einer Umgebung mit mangelhafter "Tech Hygiene" (z.\,B. unklare Zugriffsrechte, übermäßige Dateifreigaben) kann ein KI-Assistent sensible Daten unbeabsichtigt exponieren und zu massiven Datenlecks (Data Leakage) führen\footcite{MicrosoftLearn2025CopilotSecurity}.

\paragraph{Responsible-AI-Prinzipien}
Als Antwort auf diese Risiken haben sich "Responsible AI" (RAI)-Frameworks als De-facto-Standard etabliert. Ein RAI-Framework ist die Gesamtheit der Prinzipien, Richtlinien und Kontrollen, die sicherstellen, dass KI-Systeme ethisch, zuverlässig und gesetzeskonform eingesetzt werden\footcite{Nemko2025RAIFramework}. Diese Frameworks basieren typischerweise auf globalen Standards (z.\,B. OECD AI Principles, EU AI Act) und umfassen Kernprinzipien wie Fairness (Bias-Vermeidung), Transparenz (Erklärbarkeit), Sicherheit, Datenschutz und menschliche Aufsicht\footcite{Cognizant2025RAIPrinciples}. Ein solches Framework ist die Grundvoraussetzung für die Schaffung von Vertrauen und die Einhaltung von Compliance.

\subsection{Adoptions- und Akzeptanztheorien (TAM, UTAUT)}
Die bloße Bereitstellung einer Technologie garantiert nicht deren Nutzung. Die Akzeptanzforschung, insbesondere die "Unified Theory of Acceptance and Use of Technology" (UTAUT), liefert ein robustes Modell zur Erklärung, warum Individuen eine Technologie annehmen oder ablehnen\footcite{UTAUT2_Students_Cedtech2025}. UTAUT postuliert, dass die Verhaltensabsicht (\textit{Behavioral Intention}, BI) und die tatsächliche Nutzung primär von vier Kernkonstrukten abhängen:

\begin{description}
    \item[Performance Expectancy (PE) – Leistungserwartung:] Der Grad, zu dem eine Person glaubt, dass die Nutzung des Systems ihre Arbeitsleistung verbessert. Im Kontext von Copilot wird dies durch die in Kap.~2.1 genannten Produktivitätsversprechen (z.\,B. 14 Min./Tag) genährt. Die PE gilt oft als stärkster Prädiktor für die Nutzungsabsicht\footcite{UTAUT2_Students_Cedtech2025}.

    \item[Effort Expectancy (EE) – Aufwandserwartung:] Der Grad der Leichtigkeit, der mit der Nutzung des Systems verbunden ist. Bei generativer KI ist die EE eine \textit{kritische Barriere}. Die Interaktion erfordert die neue Kompetenz des "Prompt-Engineering" – die Fähigkeit, effektive Anfragen zu formulieren. Die wahrgenommene Komplexität dieser Aufgabe führt zu einer hohen kognitiven Last. Nutzerberichte zeigen, dass Frustration schnell einsetzt; Anwender geben oft schon nach wenigen erfolglosen Versuchen (ca. 15 Minuten) auf und kehren zu manuellen Methoden zurück\footcite{Arxiv2025BeyondTraining}.

    \item[Social Influence (SI) – Sozialer Einfluss:] Die Wahrnehmung, dass wichtige Bezugspersonen (Vorgesetzte, einflussreiche Kollegen) glauben, man solle das System nutzen. Die Forschung bestätigt die entscheidende Rolle von Führungskräften, die KI sichtbar selbst nutzen, sowie von "KI-Champions" oder "Power Usern"\footcite{SloanReview2025ScaleGenAI}. Diese Champions normalisieren die Nutzung, demonstrieren Anwendungsfälle und helfen, kulturellen Widerstand oder "AI Shaming" (die Wahrnehmung, KI-Nutzung sei "Betrug") zu überwinden\footcite{SloanReview2025ScaleGenAI}.

    \item[Facilitating Conditions (FC) – Unterstützende Rahmenbedingungen:] Die Überzeugung, dass eine organisationale und technische Infrastruktur zur Unterstützung der Nutzung existiert. Im Kontext von Copilot ist die wichtigste FC die Befähigung zur \textbf{"Prompt Literacy"} (Prompt-Kompetenz). Dies ist mehr als nur "Engineering"; es ist eine ganzheitliche Fähigkeit, Prompts strategisch zu formulieren, zu bewerten und zu verfeinern\footcite{Monash2025PromptLiteracy}. Mangelnde Investitionen in diese Fähigkeit führen zu einer Negativspirale: Schlechte Prompts erzeugen schlechte Ergebnisse, was die PE senkt und die EE erhöht, und letztlich zur Ablehnung des Werkzeugs führt.
\end{description}

Über die reine Akzeptanz hinaus muss auch der aktive \textit{Widerstand} betrachtet werden. Dieser ist oft tief in psychologischen Faktoren verwurzelt, darunter die Angst vor Arbeitsplatzverlust, mangelndes Vertrauen (Halluzinationen), die Furcht vor Überwachung und die Bedrohung der eigenen beruflichen Identität und Autonomie\footcite{CyberSecurityIntel2025EmployeeResistance}\footcite{Pandatron2025OvercomingResistance}.

\subsection{Change-Management-Modelle (Lewin, Kotter, ADKAR)}
Die Einführung von KI-Assistenten ist weniger ein Technologie-Rollout als vielmehr ein tiefgreifender kultureller Wandel. Während Modelle wie Lewin (Unfreeze, Change, Refreeze) oder ADKAR (Awareness, Desire, Knowledge, Ability, Reinforcement) den Prozess konzeptualisieren, bietet Kotters 8-Stufen-Modell einen praxisorientierten, sequenziellen Rahmen für die Führung dieses Wandels und dient daher als "Implementation Engine" für diese Arbeit\footcite{Prosci2025KotterTheory}.

Kotters Modell (Stufen: 1. Dringlichkeit schaffen, 2. Führungskoalition aufbauen, 3. Vision entwickeln, 4. Vision kommunizieren, 5. Hindernisse beseitigen, 6. Kurzfristige Erfolge erzielen, 7. Beschleunigung aufrechterhalten, 8. Wandel verankern)\footcite{Splunk2025Kotter8Steps} ist für die KI-Einführung besonders relevant:

\begin{itemize}
    \item \textbf{Stufe 3 (Vision entwickeln):} Die Vision muss das "Warum" der KI-Einführung definieren und die in Kap.~2.2 genannten Ängste aktiv adressieren. Das Narrativ muss auf "Befähigung und Unterstützung" statt "Ersatz und Automatisierung" fokussieren\footcite{UPC2025AIProfessionalWorld}.
    \item \textbf{Stufe 5 (Hindernisse beseitigen):} Diese Hindernisse sind nicht nur technischer Natur. Sie umfassen veraltete Prozesse (KI darf nicht auf ineffiziente Abläufe "aufgesetzt" werden), organisationale Silos und kulturellen Widerstand wie "AI Shaming"\footcite{SloanReview2025ScaleGenAI}.
\end{itemize}

Kotters Modell muss im KI-Kontext spezifisch ergänzt werden. Die Stufen 5 (Hindernisse beseitigen) und 6 (Kurzfristige Erfolge) sind direkt von der Schaffung neuer Kompetenzen abhängig. Die Fähigkeit ("Ability" im ADKAR-Modell), die Technologie erfolgreich anzuwenden, ist an die in Kap.~2.2 identifizierte "Prompt Literacy" (FC) gekoppelt. Ein Change-Management-Prozess, der kein dediziertes Training für diese Kernkompetenz vorsieht, wird den Schwung verlieren, da die Mitarbeiter nicht in der Lage sind, die "kurzfristigen Erfolge" zu generieren.

\subsection{Technologie- und Governance-Perspektive}
Die erfolgreiche Implementierung von KI-Assistenten hängt von einem robusten technologischen und organisatorischen Rahmen ab, der über die reine Lizenzvergabe hinausgeht.

\paragraph{Tech Hygiene und Compliance}
Wie in Kap.~2.1 dargelegt, ist die "Tech Hygiene" – der Zustand der Informations-Governance (Datenqualität, Zugriffsrechte) – die Achillesferse der Enterprise-KI. Das Prinzip "Garbage In, Garbage Out" (GIGO) hat hier eine doppelte Bedeutung:
\begin{enumerate}
    \item \textbf{Sicherheit (Garbage In $\rightarrow$ Leakage Out):} Unstrukturierte, übermäßig geteilte Daten führen zu Compliance-Risiken\footcite{MicrosoftLearn2025CopilotSecurity}.
    \item \textbf{Akzeptanz (Garbage In $\rightarrow$ Garbage Out):} Veraltete oder irrelevante Daten in der Wissensbasis führen zu ungenauen oder generischen KI-Antworten. Dies zerstört die Leistungserwartung (PE) und erhöht die Aufwandserwartung (EE) des Nutzers (siehe Kap.~2.2).
\end{enumerate}
Die Vorbereitung auf Copilot ist daher primär eine Initiative zur Informations-Governance. Werkzeuge wie Microsoft Purview, die Datenklassifizierung (Data Classification), Vertraulichkeitsbezeichnungen (Sensitivity Labels) und Data Loss Prevention (DLP) ermöglichen, sind technische Voraussetzungen für einen sicheren und effektiven Betrieb\footcite{MicrosoftLearn2025PurviewGenAI}.

\paragraph{Operating Model (CoE und Champions)}
Um eine unkoordinierte, "wilde" Einführung zu verhindern, die zu Risiken und doppelter Aufwandsentwicklung führt, ist eine zentrale Governance-Struktur erforderlich. Als Best Practice hat sich ein "AI Center of Excellence" (CoE) etabliert, oft in einem hybriden "Hub-and-Spoke"-Modell. Der "Hub" (das CoE) setzt Standards, verwaltet die Technologie und die RAI-Prinzipien, während die "Spokes" (Geschäftsbereiche) Anwendungsfälle identifizieren und Innovation vorantreiben\footcite{Dataiku2025CoEBestPractices}. Dieses Modell balanciert zentrale Kontrolle mit geschäftlicher Agilität. Integraler Bestandteil dieses Modells sind die "KI-Champions" (Kap.~2.2), die als Multiplikatoren und First-Level-Support in den "Spokes" agieren.

\paragraph{KPI- und Policy-Rahmen}
Die Messung des Erfolgs (ROI) von generativer KI ist komplex. Ein reiner Fokus auf quantitative KPIs wie Lizenznutzung oder Zeitersparnis greift zu kurz\footcite{Medium2025ROIPlaybook}. Ein umfassender Rahmen muss qualitative KPIs einbeziehen, wie die Verbesserung der Arbeitsqualität, Mitarbeiterzufriedenheit (Reduktion von Burnout) und Innovationsgeschwindigkeit. Fallstudien (z.\,B. Novo Nordisk) zeigen, dass die wahrgenommene Verbesserung der Arbeitsqualität oft stärker mit der Zufriedenheit korreliert als die reine Zeitersparnis\footcite{SloanReview2025ScaleGenAI}.

\subsection{Integriertes Bezugsrahmenmodell \newline(Technologie–Organisation–Mensch)}
Die vorangegangenen Abschnitte zeigen, dass die Einführung von KI-Assistenten nicht isoliert in einer Dimension betrachtet werden kann. Für die kritische Analyse in dieser Arbeit werden die Modelle zu einem integrierten T–O–M-Bezugsrahmen synthetisiert:

\begin{itemize}
    \item \textbf{T (Technologie):} Diese Dimension umfasst die technische Bereitschaft. Sie wird definiert durch die Qualität der \textbf{"Tech Hygiene"} (Kap.~2.4), die Implementierung von Sicherheits-Tools (z.\,B. Purview) und die Erfüllung technischer Voraussetzungen (z.\,B. Update-Kanäle, Lizenzen).
    \item \textbf{O (Organisation):} Diese Dimension umfasst die strategische Steuerung. Sie wird definiert durch den \textbf{Change-Prozess} (Kotter, Kap.~2.3), die Governance-Struktur (z.\,B. \textbf{CoE}), das \textbf{RAI-Framework} und die Definition von Erfolgs-KPIs (Kap.~2.4).
    \item \textbf{M (Mensch):} Diese Dimension umfasst die individuelle Adoption. Sie wird durch die Konstrukte des \textbf{UTAUT}-Modells (Kap.~2.2) definiert, insbesondere durch den Konflikt zwischen Leistungserwartung (PE) und Aufwandserwartung (EE), den sozialen Einfluss (SI) und die kritische Notwendigkeit der \textbf{"Prompt Literacy"} als wichtigste Rahmenbedingung (FC).
\end{itemize}

Die zentrale Hypothese dieses Bezugsrahmens ist die \textbf{Interdependenz der Dimensionen}. Ein Versäumnis in der T-Dimension (z.\,B. schlechte "Tech Hygiene") führt unweigerlich zu einem Scheitern in der M-Dimension (z.\,B. "Garbage In, Garbage Out" zerstört die PE und EE), selbst wenn die O-Dimension (Change-Kommunikation) perfekt ausgeführt wird. Dieses T–O–M-Modell dient als analytische Linse für die Untersuchung der Fallstudie in Kapitel 4.

% ----------------------------------------

\section{Forschungsdesign und Methodik}
\subsection{Designwahl und Forschungslogik}
Embedded Single-Case-Study (KS) mit Mixed-Methods; Ziel ist die theoriegeleitete Erklärung (analytische Verallgemeinerung), nicht Populationsschätzung.

\subsection{Datenquellen und Sampling}
\begin{itemize}
    \item \textbf{Qualitativ:} leitfadengestützte Interviews (Adoption/Programmleitung, Enablement/Training, IT, Endnutzer:innen).
    \item \textbf{Dokumente/Artefakte:} Trainings- und Meetingunterlagen, Governance-/Kommunikationsartefakte.
    \item \textbf{Quantitativ:} Nutzungs-/Lizenz-KPIs (sofern zugänglich), Feedback-Umfragen.
\end{itemize}

\subsection{Operationalisierung der Variablen}
UTAUT-Skalen (BI, PE, EE, SI, FC) kombiniert mit einem \emph{Change-Intensitätsindex} (Kommunikationsfrequenz, Trainingsvolumen, Sichtbarkeit des Sponsorings); Outcome-Proxies (aktive Nutzerquote, Nutzungsfrequenz, Feature-Breite).

\subsection{Auswertungsverfahren}
Qualitatives thematisches Kodieren (deduktiv nach T–O–M, ergänzt um induktive Subcodes), Reliabilitätschecks; deskriptive Statistik und einfache Zusammenhangsanalysen (r/β) für Hypothesen-Sondierung.

\subsection{Gütekriterien und ethische Aspekte}
Triangulation, Member-Checks, Transparenz; DSGVO-konforme Anonymisierung und sichere Datenspeicherung.

% ----------------------------------------

\section{Ergebnisse entlang der Theoriekonstrukte}
% Wichtig: keine Prozesschronik; die Vignetten illustrieren Theorieaussagen knapp.
\subsection{Technologische Dimension (T)}
Welche \emph{Tech Hygiene}-Faktoren sind notwendig (App-Parität, Update-Kanäle, Identität/Policy)? 
\paragraph{Case-Vignette C (KS) – Tech Hygiene}
Adressierte App-/Channel-Inkonsistenzen (Consumer- vs.\ Enterprise-App, Office-Kanal-Konsistenz) als notwendige Bedingung für verlässliche Nutzung.\footnote{Interne Meetingzusammenfassung, \emph{Copilot Adoption – Weekly Team Meeting}, 16.\,04.\,2025.}

\subsection{Organisationale Dimension (O)}
Sponsoring, Steering, Lizenz-Governance, Champions-Programm, KPI-Boards als Enabler struktureller Verankerung.
\paragraph{Case-Vignette D (KS) – Governance \& Champions}
Steering-Vorbereitung und Survey-Insights (häufigste Nutzung: Chat, Outlook, Teams) dienten als Grundlage für Kommunikation, Learning und Champions.\footnote{Interne Meetingzusammenfassung, \emph{Copilot Adoption – Weekly Team Meeting}, 16.\,04.\,2025.}

\subsection{Menschliche Dimension (M)}
Erwartungsmanagement, Training, Community/Office Hours, Sprach-/Kontextaspekte (z.\,B.\ Meeting-Funktionen, Transkript-Policies).
\paragraph{Case-Vignette A (KS) – Training \& Community}
Mehrsprachige Trainingsserie (EN/DE) mit Office Hours und Community-Nutzung als Beispiel für \emph{Facilitating Conditions}.\footnote{Interne Sessions, \emph{Onboarding / Plan \& Organize / Summarize \& Catch Up / Search \& Analyze / Create Content}, März–April 2025.}
\paragraph{Case-Vignette E (KS) – Responsible Use}
Pilotierung einer Meeting-Funktion (Facilitator) gekoppelt mit Transkript-Leitlinien aus Legal als \emph{Trial-before-Scale}.\footnote{Interne Meetingzusammenfassung, \emph{Copilot Adoption – Weekly Team Meeting}, 28.\,05.\,2025.}

\subsection{Zwischenfazit}
Muster: \emph{Tech Hygiene} $\rightarrow$ \emph{Enablement} $\rightarrow$ \emph{Habit Formation} unter Governance-Dach; Störfaktoren werden über Feedback-Loops adressiert.
\paragraph{Case-Vignette B (KS) – Measurement \& Lizenzen}
Aktive Nutzerzahlen und knappe Lizenzkontingente führten zu Lizenz-Review und Cleanup als Teil der Governance.\footnote{Interne Projektsitzung, \emph{Copilot Adoption – Project Closure}, 29.\,07.\,2025.}

% ----------------------------------------

\section{Diskussion}
\subsection{Theorieabgleich und Interpretation}
Abgleich der Befunde mit UTAUT (Zusammenhänge BI/PE/EE/SI/FC) und Change-Logiken (Lewin/Kotter/ADKAR); Ergänzungsbedarf bei Data-/Prompt-Literacy als eigenständiger Hebel.

\subsection{Designprinzipien und Generalisierung}
\begin{description}
    \item[DP1 – Tech Readiness:] Vor dem Scale-out \emph{Hygiene-Faktoren} schließen (App-Parität, Update-Kanal, Identity, DLP).
    \item[DP2 – Governance \& Capacity:] Lizenz-/Use-Case-Governance und eine \emph{Adoption- \& Learning-Fabrik} (Community, Office Hours, Champions, KPI-Boards).
    \item[DP3 – Change Mechanisms:] Sichtbares Sponsoring, sequenzierte Pilotierung (\emph{Trial $\rightarrow$ Scale}), „Learning in public“.
    \item[DP4 – Measurement:] Duale Metriken (\emph{Use} $\times$ \emph{Capability}), Feedback-Loops, Scorecards.
\end{description}

\subsection{Implikationen für Forschung und Praxis}
Skizze eines integrierten T–O–M$\times$UTAUT$\times$Change-Modells; Übertragbarkeit und Einbettung in Enterprise-Programme (z.\,B.\ Verzahnung mit Power Platform für wiederholbare Use-Cases).

\subsection{Limitationen der Studie}
Single-Case, eingeschränkter Datenzugang/Telemetrie, potenzielle Biases; Bedarf an Mehrfall- und Längsschnittstudien.

% ----------------------------------------

\section{Fazit und Ausblick}
\subsection{Kernaussagen}
Adoption generativer KI ist primär ein Governance- und Lernproblem – Technik ist notwendig, aber nicht hinreichend.

\subsection{Handlungsempfehlungen}
\textbf{Kurzfristig (0–3 Monate):} Rollen/Sponsoring klären, Kommunikationspaket, Pilot-Use-Cases, Sicherheits-/Policy-Check.\\
\textbf{Mittelfristig (3–12 Monate):} Skalierung, CoE/Adoption-Team, Trainingskatalog, KPI-Dashboard.\\
\textbf{Langfristig:} Kontinuierliche Verbesserung, Responsible-AI-Kontrollen.

\subsection{Ausblick auf zukünftige Forschung}
Hypothesenbündel für Mehrfall-Studien; Replikationen in unterschiedlichen Domänen/Regulatoriken.

%----------------------------------------------------------------------------------------
%   ANHANG (optional) – Instrumente und Materialien
%----------------------------------------------------------------------------------------
\newpage
\appendix
\section{Interviewleitfaden (Auszug, T–O–M)}
\subsection*{Technologie (T)}
Welche technischen Voraussetzungen erwiesen sich als Enabler/Show-Stopper (App-Parität, Update-Kanäle, Identitäts-/Policy-Themen)?\\
\subsection*{Organisation (O)}
Welche Governance-/Rollenentscheidungen (Sponsoring, Champions, KPIs) hatten spürbare Effekte?\\
\subsection*{Mensch (M)}
Wie wirkten Trainings, Community und Kommunikation auf Selbstwirksamkeit und Nutzungsgewohnheiten?

\section{Umfrageinstrument (Auszug)}
UTAUT-Items (BI, PE, EE, SI, FC; 5-Punkte-Likert) + Block \emph{Change-Intensität} (Kommunikationsfrequenz, Trainingsstunden, Sponsoring-Sichtbarkeit) + Block \emph{Adoption-Meetings} (Nützlichkeit, Gründe).

\section{Kodierleitfaden und zusätzliche Tabellen}
Deduktive Hauptcodes (T–O–M), induktive Subcodes; Beispiel-Codings, Reliabilitätsübersicht (falls durchgeführt).

% \newpage
% \appendix
% \section{Zusätzliche Daten}
% \addcontentsline{toc}{section}{Anhang A: Zusätzliche Daten} % Anhang ins Inhaltsverzeichnis
% \lipsum[16]
% \begin{figure}[htbp] 
%     \centering
%     \rule{0.5\textwidth}{0.3\textwidth}
%     \caption{Abbildung im Anhang}
%     \label{fig:anhangbild}
%     \ownsource
% \end{figure}

%----------------------------------------------------------------------------------------
%   LITERATURVERZEICHNIS
%----------------------------------------------------------------------------------------
\newpage
\setstretch{1.0} % Einfacher Zeilenabstand für Literaturverzeichnis
\printbibliography[heading=bibintoc, title=Literaturverzeichnis] % heading=bibintoc fügt es zum ToC hinzu

%----------------------------------------------------------------------------------------
%   SELBSTSTÄNDIGKEITSERKLÄRUNG
%----------------------------------------------------------------------------------------
\newpage
\section*{Selbstständigkeitserklärung}
\addcontentsline{toc}{section}{Selbstständigkeitserklärung} % Ins Inhaltsverzeichnis

\setstretch{1.5} % Zurück zum normalen Zeilenabstand für diesen Text
\noindent
Ich versichere hiermit, dass ich die vorliegende Arbeit mit dem Thema: \textit{Beispielhafter Titel einer wissenschaftliche Arbeit im Nominalstil} selbstständig verfasst und keine anderen als die angegebenen Quellen und Hilfsmittel benutzt habe. Ich versichere zudem, dass die eingereichte elektronische Fassung mit der gedruckten Fassung übereinstimmt.

\vspace{2cm}
\noindent
Spaichingen,den \today
\hfill % Sorgt dafür, dass die Minipage nach rechts geschoben wird
\begin{minipage}[t]{5cm} % [t] für top alignment
    \centering % Zentriert den Inhalt der Minipage
    % Ersetzen Sie signature.jpg durch den Pfad zu Ihrer Unterschriftsdatei
    % Wenn Sie keine Bilddatei haben, kommentieren Sie die nächste Zeile aus
    % \includegraphics[width=\linewidth]{signature.jpg}\\[-5ex] % Breite auf Linienbreite, Abstand angepasst
    \rule{\linewidth}{0.4pt}\\[1ex] % Linie unter der Unterschrift
    Jamie Adler
\end{minipage}

\end{document}
